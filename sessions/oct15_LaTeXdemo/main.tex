\documentclass{article} % but also try: {artikel3}
% IEEEtrans
% SIAMwhatever
\usepackage[utf8]{inputenc}
\usepackage{graphicx,pslatex}
\input list

\title{SDS 335 Demo}
\author{Victor Eijkhout }
\date{15 October 2020}

%\textwidth=2in
\begin{document}

\maketitle

\tableofcontents

\section{Introduction}

This is the first section of my document.
It has two sentences.

After a blank line
it
starts a new paragraph.
This paragraph was indented. Therefore it looked
weird when it was so short.
% this is a comment: \[ indicates a display formula
\[ \left( x \right ) \left( \int_0^\infty x\partial x= 5 \right) \]
Mathematics is clever $\int_0^\infty x\partial x= 5$
% this is a comment: $foo$ is an in-line formula
because it knows where it goes.

\section{Code}

Verbatim mode uses mono-space
\begin{verbatim}
// first a function
void f() { return 5;}
/* now the main */
int main() {
  return 0;
}
\end{verbatim}
But the listing package is much nicer:
\lstset{language=C,style=reviewcode}
\begin{lstlisting}
// first a function
void f() { return 5;}
/* now the main */
int main() {
  return 0;
}
\end{lstlisting}
\section{Recapitulation}
\label{sec:recap}
\subsection{foo}
\subsubsection{bar}

Final thoughts, to elaborate on section \ref{sec:dev}.
\begin{itemize}
    \item first
    \item second
\end{itemize}

In this figure~\ref{fig:butterfly} we see.

\begin{figure}[ht] 
%"ht" means "here if possible, otherwise top of next page
    \centering
\includegraphics[scale=.55]{butterfly}
    \caption{Caption}
    \label{fig:butterfly}
\end{figure}

\section{Development}
\label{sec:dev}

\begin{enumerate}
    \item one thought
    \item another
\end{enumerate}

For related work see~\cite{Victor:otherbook}.
Another paragraph.

%\bibliographystyle{plain} % IEEEtrans
%\bibliography{vle}

\end{document}
